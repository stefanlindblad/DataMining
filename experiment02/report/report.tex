\documentclass[12pt,a4paper]{scrartcl}
\usepackage[utf8]{inputenc}
\usepackage{graphicx}
\usepackage{ngerman}
\usepackage{url}
\usepackage{amsmath}
\usepackage{caption}
\usepackage{wrapfig}
\usepackage{eurosym}
\usepackage{biblatex}
\usepackage{url}
\usepackage{color}
\usepackage{listings}
\usepackage{hyperref}
\usepackage[table]{xcolor}
\linespread{1.4}

\definecolor{mygreen}{rgb}{0,0.6,0}
\definecolor{mygray}{rgb}{0.5,0.5,0.5}
\definecolor{mylightgray}{rgb}{0.7,0.7,0.7}
\definecolor{mylightergray}{rgb}{0.9,0.9,0.9}
\definecolor{mymauve}{rgb}{0.58,0,0.82}

\let\origitemize\itemize
\def\itemize{\origitemize\itemsep0pt}

\lstset{ 
  backgroundcolor=\color{white},   
  basicstyle=\ttfamily\footnotesize,          
  breakatwhitespace=false,         
  breaklines=true,  
  commentstyle=\color{mygreen}, 
  escapeinside={\%*}{*)}, 
  extendedchars=true,             
  keepspaces=true,                 
  keywordstyle=\color{blue},
  language=Octave,
  numbers=left,                   
  numbersep=15pt,                  
  numberstyle=\tiny\color{mygray}, 
  showspaces=false,                
  showstringspaces=false,          
  showtabs=false,                  
  stringstyle=\color{mymauve},
  tabsize=2,
  title=\lstname,
  captionpos=b
}

\renewcommand*\lstlistingname{Codebeispiel}    %Rename Listings

\renewcommand*\thesection{\arabic{section}}

\makeatletter
\renewcommand\subparagraph{\@startsection{subparagraph}{5}{\parindent}%
    {3.25ex \@plus1ex \@minus .2ex}%
    {0.75ex plus 0.1ex}% space after heading
    {\normalfont\normalsize\bfseries}}
\makeatother

\begin{document}
\title{Praktikum Data Mining}
\subtitle{Recommender Systems}
\author{Oliver Fesseler \and Maria Florus\ss \and Stefan Seibert \and  Daniel Grie\ss haber}
\maketitle
\newpage

\part*{ Durchf\"uhrung 1: Fiktive Filmbewertung}

\section*{\"Ahnlichkeitsbestimmung}
\subparagraph{Bedeutung des \"Ubergabeparameters \lstinline{normed} in der Funktion \lstinline{sim_euclid}:}

Wenn der Parameter \lstinline{normed} auf \lstinline{True} gesetzt ist, wird daf\"ur gesorgt, dass die \"Ahnlichkeit normalisiert wird. 
Ohne die Normalisierung w\"urden zwei Personen sich immer un\"ahnlicher mit einer steigenden Anzahl an Filmen, die beide bewertet haben, da die Differenzen zwischen den Bewertungen einfach nur zusammengez\"ahlt werden. 
Durch die Normalisierung wird dies behoben. 

\subparagraph{Implementierung der Funktion \lstinline|topMatches(prefs, person, similarity)|:}

\begin{lstlisting}[language=Python]
def topMatches(prefs, person, similarity):

    sim = {}
    for candidate in prefs:
        if candidate == person:
            continue
        sim[candidate] = similarity(prefs, person, candidate)

    return sorted(sim.iteritems(), key=operator.itemgetter(1), reverse=True)
\end{lstlisting}

\lstinline{topMatches()} berechnet die \"Ahnlichkeit f\"ur jede in \lstinline{prefs} aufgelistete Person die \"Ahnlichkeit zu \lstinline{person} und schreibt diese in ein Dictionary (Person : \"Ahnlichkeit). Dieses wird in eine absteigend sortierte Liste von Tupeln (Person, \"Ahnlichkeit) geschrieben, welche zur\"uck gegeben wird.  

Bei der Implementierung von \lstinline{topMatches()} fiel uns auf, dass man als Similarity-Funktion nur \lstinline{sim_euclid} und \lstinline{sim_pearson} angeben kann, jedoch so keine Unterscheidung zwischen der normalisierten und der unnormalisierten Version von \lstinline{sim_euclid} gemacht werden kann.
Deshalb entschieden wir uns eine weitere Similarity-Funktion zur Verf\"ugung zu stellen: 

\begin{lstlisting}[language=Python]
def sim_euclid_normed(prefs, person1, person2):
    return sim_euclid(prefs, person1, person2, normed=True)
\end{lstlisting}

\lstinline{sim_euclid_normed} ruft einfach die urspr\"ungliche Version von \lstinline{sim_euclid} mit \lstinline{normed = True} auf. 



\subparagraph{Vergleich der \"Ahnlichkeitsma\ss e:}

Die Pearson-\"Ahnlichkeit eignet sich f\"ur pers\"onliche Wertungen besser, da Nutzer unterschiedliche Wertungsgewohnheiten haben. So bewerten einige Nutzer Filme allgemein schlechter, andere geben kontrastreichere Bewertungen f\"ur gute und schlechte Filme ab. Die Pearson-\"Ahnlichkeit gleicht diese pers\"onlichen Pr\"aferenzen aus.


\section*{Berechnung von Empfehlungen mit Userbasiertem Collaborative Filtering (UCF)}
\subparagraph{Implementierung der Funktion \lstinline{getRecommendations(prefs,person,similarity)}:}
\begin{itemize}
\item Anmerkung: Die Erkl\"arung der Funktion bezieht sich auf das Beispiel mit den Filmbewertungen.
\end{itemize}


Wie in \lstinline{topMatches()}  wird zun\"achst ein Dictionary angelegt, dass jeder Person au\ss er der `Zielperson' einen \"Ahnlichkeitswert zuordnet.

Nun wird f\"ur jede andere Person festgestellt, welche Filme diese bewertet hat, die die `Zielperson' noch nicht bewertet hat. Die Bewertung jedes Films wird in ein Dictionary geschrieben, dass jedem Film, den die `Zielperson' noch nicht gesehen hat, die Summe aller Bewertungen dieses Films zuordnet.
Gleichzeitig wird f\"ur jeden Film die Korrelation der Person, die ihn bewertet hat in einem Dictionary \lstinline{kSum} festgehalten. So ordnet \lstinline{kSum} jedem Film die Summe aller relevanten Korrelationen zu.
Zuletzt wird f\"ur jeden Film die Summe aller Bewertungen durch die Summe aller relevanten Korrelationen geteilt um den tats\"achlichen Empfehlungswert zu erhalten.
Zur\"uckgegeben wird eine absteigend sortierte Liste von Tupeln(Empfehlungswert, Film).



Beim Testen unserer Funktionen \lstinline{getRecommendations()} stellten wir fest, dass die Similarity-Funktion \lstinline{sim_pearson} nicht funktioniert, wenn eine Person nur einen Film gesehen hat, da hier die Standartabweichung berechnet wird und dies f\"ur nur einen Wert immer 0 liefert. So werden alle Korrelationen mit 0 angegeben. 
Um dies zu umgehen bieten sich die Similarity-Funktionen \lstinline{sim_euclid} oder \lstinline{sim_euclid_normed} an.


\section*{Berechnung von Empfehlungen mit Itembasiertem Collaborative Filtering (ICF)}
\subparagraph{Transponieren der \lstinline{prefs}-Matrix:}
F\"ur die Umsetzung des ICF musste die bisher verwendete Bewertungsmatrix in ihre transponierte Form \"uberf\"uhrt werden. Dazu wurde die Funktion \lstinline{transposeMatrix(prefs) } implementiert. Diese iteriert zun\"achst \"uber alle User und schreibt die bewerteten Filme in ein Dictionary. Danach sind alle Filme die ein User bewertet haben kann im Dictionary vorhanden. Nun werden alle Filmbewertungen aus der urspr\"unglichen Matrix extrahiert und in der neuen Matrix eingef\"ugt.
Die transponierte Matrix wird zur\"uck gegeben.

Die Transformation ist notwendig, da die Funktionen aus den oberen Aufgaben unver\"andert wiederverwertet werden sollten. 

\subparagraph{Implementierung der Funktion \lstinline{calculateSimilarItems()}:}
Die Funktion stellt fest wie \"ahnlich sich die Filme untereinander sind und gibt f\"ur jeden Film ein Dictionary zur\"uck in dem die \"Ahnlichkeit zu allen anderen Filmen festgehalten ist.

\subparagraph{Implementierung der Funktion \lstinline{getRecommendedItems(prefs, name, similar_items)}:}
Diese Funktion soll durch ein ICF, einem User Filme die er noch nicht gesehen hat empfehlen. Diese Empfehlung basiert auf der Bewertung der bisher gesehenen Filme. 

Daf\"ur werden zun\"achst alle Filme, die ein User bewertet hat in eine Liste geschrieben. Danach werden alle Filme die der User nicht bewertet hat ebenfalls in eine Liste geschrieben. 
Nun wird die \"Ahnlichkeit zwischen allen Filmen mit der Funktion \lstinline{calculateSimilarItems()} berechnet. \"Uber die nicht bewerteten Filme wird nun iteriert und die \"Ahnlichkeit zu jedem vom User bewerteten Film ausgelesen. Diese \"Ahnlichkeit wird nun mit der Bewertung des Users multipliziert um eine userspezifische Gewichtung der \"Ahnlichkeit zu garantieren. Nun wird f\"ur jeden Film die Summe \"uber diese gewichtete \"Ahnlichkeiten durch die Summe der ungewichteten \"Ahnlichkeiten geteilt. Dieser Wert beschreibt wie empfehlenswert der Film f\"ur den aktuellen User ist.
Die Funktion gibt eine Liste von Tupeln (Film, Empfehlungswert)  zur\"uck. 

\part*{Durchf\"uhrung 2: \\last.fm Musikempfehlungen}

\subsection*{Durchf\"uhrung}

Um die gew\"unschte last.fm Musikempfehlung durchf\"uhren zu k\"onnen wurde im File \lstinline{pylastRecTest.py} der Zugriff auf die last.fm API implementiert.
Hierf\"ur stand bereits die Datei \lstinline{pylast.py} zur Verf\"ugung. Nach Aufruf des Networkobjekts wurden die 10 Topfans der Band `Soulfly' abgerufen.
Anschlie\ss end die zur\"uckgelieferten Userobjekte auf deren Usernamen reduziert. 

Die beiden Methoden \lstinline{createLastfmUserDict(fanNames)} und die oben beschriebene\\ \lstinline{getRecommendations(pref, person, similarity)} wurden im File \lstinline{recommendations.py} implementiert und lediglich nach einem entsprechenden Import ausgef\"uhrt.

\lstinline{createLastfmUserDict(fanNames)} liefert dabei eine Matrix von Usernames und Bands, die von mindestens einem User geh\"ort werden, zur\"uck, wobei der Value = 1 ist, falls der einzelne User die Band h\"ort, anderenfalls 0.

Problematiken:
\begin{itemize}
\item Bei der Zuweisung aller Bands zu einem bestimmten User muss darauf geachtet werden, dass das Dictionary aller Bands (lstinline{allBands})  als deep-copy \"ubergeben wird. Anderenfalls werden die Zuweisungspaare `User - Band geh\"ort' mit jedem User f\"ur alle User \"uberschrieben, sodass die resultierte Matrix nur 1en enth\"alt.
\item die Bandnamen sollten nur aus unicode-Zeichen bestehen, da sonst einige Bands doppelt aufgef\"uhrt werden. Allerdings kann es dadurch zu Darstellungsproblemen kommen. 
\end{itemize}

Feststellungen:
\begin{itemize}
\item Nur weil ein User in den Top Fans einer Band ist, kann daraus kein R\"uckschluss auf die Platzierung der Band in seinen pers\"onlichen Top Bands gezogen werden.
\end{itemize}



\end{document}